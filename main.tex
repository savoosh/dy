\documentclass[a4paper]{article}
\usepackage[russian]{babel}
\usepackage{mathtext}
\usepackage{amsmath}

\title{Лабораторная работа №2}
\author{Заговальский Глеб}
\date{Май 2022}

\begin{document}

\maketitle

\section*{Вариант 1 \newline 1) Указать тип и метод интегрирования ур-ий}
\subsection*{a)}
\begin{equation*}
(2xy^2+3x^2+\frac{1}{x^2}+\frac{3x^2}{y^2}) dx+(2x^2y+3y^2+\frac{1}{y^2}-\frac{2x^3}{y^3}) dy = 0
\end{equation*}
\subsubsection*{Тип: уранение в полных дифференциалах}

\begin{equation*}
\frac{\partial P}{\partial y}=4xy-\frac{6x^2}{y^3}
\end{equation*}
\begin{equation*}
\frac{\partial Q}{\partial x}=4xy-\frac{6x^2}{y^3}
\end{equation*}
\begin{equation*}
\frac{\partial P}{\partial y}=\frac{\partial Q}{\partial x}
\end{equation*}

\subsubsection*{Метод:}
\begin{equation*}
\int_{x_0}^{x}\left(2xy^2+3x^2+\frac{1}{x^2}+\frac{3x^2}{y^2}\right)dx+\int_{y_0}^{y}\left(2x_0^2y+3y^2+\frac{1}{y^2}-\frac{2x_0^3}{y^3}\right)dy=u(x,y)
\end{equation*}

\subsection*{б)}
\begin{equation*}
xy(1+y^2)dx} + (1+x^2)dy = 0
\end{equation*}
\subsubsection*{Тип: уранение с разделяющимися переменными}

\subsubsection*{Метод:}
\begin{equation*}
\int \frac{xdx}{1+x^2}=-\int \frac{dy}{y(1+y^2)}
\end{equation*}

\subsection*{в)}
\begin{equation*}
2xy dx} - (x^2 - y^2)dy = 0
\end{equation*}
\subsubsection*{Тип: однородное уранение}

\subsubsection*{Метод:}
замена перемнных
\begin{gather*}
\begin{cases}
\frac{y}{x}=u \\
x=x
\end{cases}\\
y=ux \\
dy=xdu+udx
\end{gather*}

\begin{gather*}
2x^2udx-(x^2-u^2x^2)(xdu+udx)=0 \\
2x^2udx-x^3du-x^2udx+u^2x^3du+u^3x^2dx=0 \\
(x^2u+u^3x^2)dx+(-x^3+u^2x^3)du=0 \quad |:x^2 \\
x'(u+u^3)+x(-1+u^2)=0
\end{gather*}

Получили линейное по x

\begin{gather*}
\frac{dx}{du}(u+u^3)+x(-1+u^2)=0\\
\int\frac{dx}{du}=-\int\frac{(u^2-1)du}{u+u^3}
\end{gather*}

\subsection*{г)}
\begin{equation*}
y^, ctg(x) - y = 2 cos^2(x) ctg(x)
\end{equation*}
\subsubsection*{Тип: линейное по у уранение }

\subsubsection*{Метод:}
метод Лагранжна

\subsection*{д)}
\begin{equation*}
x^2y^2y^, + xy^3 = a^2
\end{equation*}
\subsubsection*{Тип: уранение Бернули по y}

\subsubsection*{Метод:}
замена переменных
\begin{gather*}
u=y^3\\
du=3y^2dy
\end{gather*}

\subsection*{е)}
\begin{equation*}
y^, = 4y^2 - 4x^2y + x^4 + x + 4
\end{equation*}
\subsubsection*{Тип: уранение Рикатти}

\subsubsection*{Метод:}
не решается в общем случае

\subsection*{ж)}
\begin{equation*}
y x^, - 2x + y^2 = 0
\end{equation*}
\subsubsection*{Тип: линейное по x уравнение }

\subsubsection*{Метод:}
метод Лагранжа

\section*{Вариант 1 \newline 2) Проинтегрировать ур-ие, установив вид интегрирующего множителя}
\begin{equation*}
(x^3+x^3\ln{x}+2y)dx+(3y^2x^3-x)dy=0
\end{equation*}
\begin{equation*}
\Psi(y): \quad \frac{Q'_x-P'_y}{P}=
\frac{9x^2y^2-3}{x^3+x^3\ln(x)+2y}=f(x,y) \Rightarrow \Psi\neq\Psi(y)
\end{equation*}
\begin{equation*}
\Psi(x): \quad \frac{P'_y-Q'_x}{Q}=
\frac{3-9x^2y^2}{3x^3y^2-x}=
\frac{3(1-3x^2y^2)}{-x(1-3x^2y^2)}=-\frac{3}{x}
\Rightarrow \Psi=\Psi(x)
\end{equation*}
\begin{equation*}
\mu(x)=e^{\int\Psi(x)dx}=e^{-\int\frac{3}{x}dx}=e^{-3\ln x}=x^{-3}
\end{equation*}
\begin{equation}
\left(1+\ln(x)+\frac{2y}{x^3}\right)dx+\left(3y^2+-\frac{1}{x^2}\right)dy=0 \label{pd},\quad x\neq0
\end{equation}
\begin{equation*}
\frac{\partial P_1}{\partial y}=\frac{\partial Q_1}{\partial x}=\frac{2}{x^3}
\end{equation*}
(\ref{pd}) - уравнение в полнных дифференциалах
\begin{equation*}
\frac{\partial u}{\partial x} = 1+\ln x+\frac{2y}{x^3}
\end{equation*}
\begin{equation*}
u(x,y)= \int \left(1+\ln x+\frac{2y}{x^3}\right)dx
\end{equation*}
\begin{equation*}
u(x,y)= x\ln x-\frac{y}{x^2}+c(y)
\end{equation*}
\begin{equation*}
\frac{\partial u}{\partial y} = -\frac{1}{x^2}+c'(y)
\end{equation*}
\begin{equation*}
c'(y)=3y^2
\end{equation*}
\begin{equation*}
c(y)=y^3
\end{equation*}
\begin{equation*}
c=x\ln x-\frac{y}{x^2}+y^3
\end{equation*}
\begin{equation*}
x=0 \Rightarrow u'_xdx+u'_ydy=0 \Rightarrow x=0\text{ - решение.}
\end{equation*}
Ответ: $c=x\ln x-\frac{y}{x^2}+y^3$,
$x=0$.

\section*{Вариант 1 \newline 3) Преобразовать уравнение с помощью подстановки, указать тип и метод интегрирования полученного ур-ия}
\begin{equation*}
y(x+\ln y)+(x-\ln y)y'=0 \quad \ln y = \eta \quad d\eta=\frac{dy}{y}
\end{equation*}
\begin{equation*}
\frac{dy}{d\eta}(x+\eta)+(x-\eta)\frac{dy}{dx}=0 \quad |:dy
\end{equation*}
\begin{equation*}
(x+\eta)dx+(x-\eta)d\eta=0
\end{equation*}
\begin{equation*}
\frac{\partial P}{\partial \eta}=\frac{\partial Q}{\partial x}=1 \Rightarrow \text{уравнение в полных дифференциалах}
\end{equation*}
метод решения:
\begin{equation*}
\int_{x_0}^{x}(x+\eta)dx+\int_{\eta_0}^{\eta}(x-\eta)d\eta=u(x,y)
\end{equation*}
\begin{equation*}
\frac{x^2}{2}+x\eta-\frac{\eta^2}{2}=c
\end{equation*}
\begin{equation*}
\frac{x^2}{2}+x\ln y-\frac{\ln^2{y}}{2}=c
\end{equation*}
Ответ:$\frac{x^2}{2}+x\ln y-\frac{\ln^2{y}}{2}=c$
\section*{Вариант 1 \newline 4) Решить задачу Коши}
\begin{equation*}
dy=(y-2)^{\frac{2}{3}}dx,\quad y|_{x=1}=2
\end{equation*}
\begin{equation*}
d(y-2)(y-2)^{-\frac{2}{3}}=dx, \quad y-2=t
\end{equation*}
\begin{equation*}
\int t^{-\frac{2}{3}}dt=\int dx
\end{equation*}
\begin{equation*}
3t^{\frac{1}{3}}=x+c
\end{equation*}
\begin{equation*}
3(y-2)^{\frac{1}{3}}=x+c
\end{equation*}
\begin{equation*}
0=1+c \Rightarrow c=-1
\end{equation*}
\begin{equation*}
3(y-2)^{\frac{1}{3}}=x-1
\end{equation*}
Ответ: $3(y-2)^{\frac{1}{3}}=x-1$.

\section*{Вариант 2 \newline 1) Указать тип и метод интегрирования ур-ий}
\subsection*{a)}
\begin{equation*}
xy(1 + y^2)dx - (1 + x^2)dy = 0
\end{equation*}
\subsubsection*{Тип: уранение с разделяющимися переменными}

\subsubsection*{Метод:}
\begin{equation*}
\frac{x}{1 + x^2}dx - \frac{1}{y(1+y^2)}dy = 0
\end{equation*}

\subsection*{б)}
\begin{equation*}
\frac{(x)^2}{(x - y)^2}dy - \frac{(y)^2}{(x - y)^2}dx = 0
\end{equation*}
\subsubsection*{Тип: однородное уравнение степени 0}

\subsubsection*{Метод:}
\begin{equation*}
\frac{x^2dy - y^2dx}{(x - y)^2} = 0
\end{equation*}
\begin{equation*}
\frac{(tx)^2}{(tx - ty)^2}dy - \frac{(ty)^2}{(tx - ty)^2}dx = 0
\end{equation*}
\begin{equation*}
\frac{t^2x^2}{t^2(x - y)^2}dy - \frac{t^2y^2}{(x - y)^2t^2}dx = 0
\end{equation*}

\subsection*{в)}
\begin{equation*}
(x^2+ y^2)dx - 2xy = 0
\end{equation*}
\subsubsection*{Тип: однородное уравнение степени 2}

\subsubsection*{Метод:}
\begin{equation*}
(x^2+ y^2)dx - 2xy = 0
\end{equation*}
\begin{equation*}
((xt)^2+ (yt)^2)dx - 2(tx)(ty) = 0
\end{equation*}

\subsection*{г)}
\begin{equation*}
(4 - x^2)y^, + xy = 4
\end{equation*}
\subsubsection*{Тип: линейное по y уравнение}

\subsubsection*{Метод:}
\begin{equation*}
y' + \frac{x}{4 - x^2}y = 4
\end{equation*}

\subsection*{д)}
\begin{equation*}
y' tg(x) + 2y tg(x)^2 = a y^2
\end{equation*}
\subsubsection*{Тип: уравнение Бернули}

\subsubsection*{Метод:}
\begin{equation*}
y'\tg(x) + 2y\tg(x)^2 = a y^2
\end{equation*}

\subsection*{е)}
\begin{equation*}
x y' = x^2 y^2 - y + 1
\end{equation*}
\subsubsection*{Тип: уравнение Риккати}

\subsubsection*{Метод:}
\begin{equation*}
y' + \frac{1}{x}y = xy^2 + \frac{1}{x}
\end{equation*}

\subsection*{ж)}
\begin{equation*}
dx + (x + y^2)dy = 0
\end{equation*}
\subsubsection*{Тип: линейное по x}

\subsubsection*{Метод:}
\begin{equation*}
x' + x = -y^2
\end{equation*}

\section*{Вариант 2 \newline 2) Проинтегрировать ур-ие, установив вид интегрирующего множителя}

\begin{equation*}
y^2(x - y)dx + (1 - xy^2)dy = 0
\end{equation*}

\begin{equation*}
\int{y^2(x - y)dx} + \int{(1 - xy^2)dy} = C
\end{equation*}

\begin{equation*}
x^2 - 2xy - \frac{2}{y} = C
\end{equation*}

\begin{equation*}
y = 0
\end{equation*}

\begin{equation*}
\mu = \mu(y): \frac{Q'_x - P'_y}{P} = \frac{-y^2 - (2xy - 3y^2)}{y^2(x - y)}
\end{equation*}

\begin{equation*}
\mu = \mu(x): \frac{P'_y - Q'_x}{Q} = \frac{2xy - 3y^2 - y^2}{1 - xy^2} = \frac{2y(x - 2y)}{1 - xy^2}
\end{equation*}

\begin{equation*}
\mu' = \frac{2}{y^2}\mu
\end{equation*}

\begin{equation*}
\frac{d\mu}{\mu} = - \frac{2dy}{y^2}
\end{equation*}

\begin{equation*}
\ln{\mu} =
\ln{y^{-2}}
\end{equation*}

\begin{equation*}
\mu = y^{-2}
\end{equation*}

Ответ: $ x^2 - 2xy - \frac{2}{y} = C, y = 0$

\section*{Вариант 2 \newline 3) Преобразовать уравнение с помощью подстановки, указать тип и метод интегрирования полученного ур-ия}

\begin{equation*}
y' = x + e^{x + 2y}
\end{equation*}

\begin{equation*}
\eta = e^{-2y}
\end{equation*}

\begin{equation*}
d\eta = -2e^{-2y}dy
\end{equation*}

\begin{equation*}
\frac{dy}{dx} = x + e^{x + 2y}
\end{equation*}

\begin{equation*}
e^{-2y}dy = (xe^{-2y} + e^x)dx
\end{equation*}

\begin{equation*}
-\frac{1}{2}d\eta = (x\eta + e^x)dx
\end{equation*}

\begin{equation*}
\eta' + 2x\eta = -2e^x \text{ - линейное по } \eta
\end{equation*}

Ответ: $\eta' + 2x\eta + ae^x = 0$

\section*{Вариант 2 \newline 4) Решить задачу Коши}

\begin{equation*}
dy = x\sqrt{y} dx,y|_{x=1} = 0
\end{equation*}

\begin{equation*}
xdx = \frac{dy}{\sqrt{y}}
\end{equation*}

\begin{equation*}
\frac{x^2}{2} = 2\sqrt{y} + C
\end{equation*}

\begin{equation*}
C = \frac{1}{2}
\end{equation*}

\begin{equation*}
x^2 - 4\sqrt{y} = 1
\end{equation*}

Ответ: $x^2 - 4\sqrt{y} = 1$

\end{document}